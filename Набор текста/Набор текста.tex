\documentclass{article}

\usepackage[utf8]{inputenc}
\usepackage[T2A]{fontenc}
\usepackage[russian]{babel}

\frenchspacing
\righthyphenmin=2

\begin{document}

\begin{center}
В этом файле содержатся команды, необходимые для набора текста
\end{center}

\textbf{1. Вывод символов}

Все буквенно-цифровые символы, содержащиеся в <<стандартной>> раскладке ASCII, в \TeX\ вводятся в текстовом режиме непосредственно, за исключением специальных символов, перечисленных ниже
\begin{center}
 \verb"{  }  $  &  #  %  _  ^  ~  \"
\end{center}

Чтобы напечатать эти символы, используются следующие команды.

\begin{center}
\begin{tabular}{|c|l|}
\hline
\textbf{Результат} & \textbf{Набираем}     \\ \hline
\{                 & \verb"\{"             \\
\}                 & \verb"\}"             \\
\$                 & \verb"\$"             \\
\&                 & \verb"\&"             \\
\#                 & \verb"\#"             \\
\%                 & \verb"\%"             \\
\_                 & \verb"\_"             \\
\^{}               & \verb"\^{}"           \\
\~{}               & \verb"\~{}"           \\
\textbackslash     & \verb"\textbackslash" \\ \hline
\end{tabular}
\end{center}

Особый статус имеет символ <<@>>: в обычном режиме он является <<обычным>> символом, однако он может также использоваться как специальный символ, как правило --- при определении новых команд.

Кроме того, часто используются дополнительные возможности, предоставляемые \TeX ом: простановка диакритических знаков, <<специальных>> символов и много другого.

\begin{center}
\begin{tabular}{|c|l||c|l|}
\hline
\textbf{Результат} & \textbf{Набираем}      & \textbf{Результат}  & \textbf{Набираем}        \\ \hline
\`e                & \verb"\`e"             &   \'e               & \verb"\'e"               \\
\^e                & \verb'\^e'             &   \~e               & \verb"\~e"               \\
\=e                & \verb"\=e"             &   \.e               & \verb"\.e"               \\
\u{e}              & \verb"\u{e}"           &   \v{e}             & \verb"\v{e}"             \\
\H{e}              & \verb"\H{e}"           &   \"e               & \verb|\"e|               \\
\c{e}              & \verb"\c{e}"           &   \d{e}             & \verb"\d{e}"             \\
\b{e}              & \verb"\b{e}"           &   \r{e}             & \verb"\r{e}"             \\
\t oo              & \verb"\t oo"           &   \ss               & \verb"\ss"               \\
\i                 & \verb"\i"              &   \j                & \verb"\j"                \\
\ae                & \verb"\ae"             &   \oe               & \verb"\oe"               \\
\textexclamdown    & \verb"\textexclamdown" &   \textquestiondown & \verb"\textquestiondown" \\ \hline
\end{tabular}
\end{center}

\begin{center}
\begin{tabular}{l@{\quad}l}
Ch\'erie, \c{c}a ne me pla\^\i t pas! & \verb"Ch\'erie, \c{c}a ne me pla\^\i t pas!" \\
K\"onigsberg, Ostpreu\ss{}en          & \verb|K\"onigsberg, Ostpreu\ss{}en|          \\
\L\'od\'z, Dav\'\i dek \v{S}imurda    & \verb"\L\'od\'z, Dav\'\i dek \v{S}imurda"    \\
\textexclamdown Besa me mucho!        & \verb"\textexclamdown Besa me mucho!"        \\
\end{tabular}
\end{center}

Еще некоторые символы, которые встречаются довольно часто:

\begin{center}
\begin{tabular}{|c|l|}
\hline
\textbf{Результат}            & \textbf{Набираем}       \\ \hline
\copyright                    & \verb"\copyright"       \\
\dots                         & \verb"\dots"            \\
\ldots                        & \verb"\ldots"           \\
\S                            & \verb"\S"               \\
\pounds                       & \verb"\pounds"          \\
неразрывный пробел            & \verb"~"                \\ \hline
\end{tabular}
\end{center}

\vspace{1cm}

\textbf{2. Дефис, короткое тире, длинное тире, минус}

\bigskip

Полиграфисты различают минимум 9 разных <<чёрточек>>: дефисоминус, дефис, минус, цифровая чёрточка, символ переноса, чёрточный буллит, короткое тире, длинное тире, горизонтальная черта.

\begin{center}
\begin{tabular}{|l|l|l|l|}
\hline
\textbf{Название}&\textbf{Набор}&\textbf{Пробелы}     &\textbf{Пример}       \\ \hline
дефис            & \verb"-"     & нет                 & кто-нибудь           \\
                 &              & (левое переносится, &                      \\
                 &              & правое нет)         &                      \\ \cline{2-4}
                 & \verb|"~|    & нет                 & юго"~запад           \\
                 &              & (оба не переносятся)&                      \\ \cline{2-4}
                 & \verb|"=|    & нет                 & Соловьев"=Седой      \\
                 &              & (оба переносятся)   &                      \\ \hline
короткое тире    & \verb"--"    & нет                 & 5--7 яблок           \\
                 &              & есть                & пять -- семь яблок   \\ \hline
длинное тире     & \verb!"---!  & есть                & Москва "--- столица  \\
                 &              &(левый неразрывный)  &                      \\ \cline{2-4}
                 & \verb!---!   & есть                & формула              \\
                 &              &(левый разрывный)    & Ньютона --- Лейбница \\ \hline
минус            & \verb"$-$"   & не важно            & $\sqrt{x} - y$       \\ \hline
\end{tabular}
\end{center}


Вообще говоря, если писать не цифрами, а словами, то следует различать по написанию следующие случаи:
\begin{itemize}
\item <<порежь три-четыре апельсина>> в смысле <<то ли одно, то ли другое>> "--- ставится дефис без пробелов;
\item <<справлюсь за десять -- пятнадцать дней>> в смысле <<в диапазоне от 10 до 15 дней>> --- ставится короткое тире с пробелами.
\end{itemize}

Однако тут не все однозначно. Как, например, трактовать фразу <<Дай мне два-три рубля>>? Интересно, 2 рубля 50 копеек годится? В таких случаях рекомендуется не влезать в тонкости и ставить дефис.

\newpage

\textbf{3. Виды шрифтов}

Для изменения вида шрифта могут использоваться как команды"=переключатели, так и команды с аргументами.

\begin{center}
\noindent\begin{tabular}{|c||l|l|}
\hline
\bfseries Эффект          & \bfseries Переключатель  &  \textbf{С аргументом}\\ \hline \multicolumn{3}{|c|}{\bf Выделение}                                                        \\ \hline
\underline{Подчеркивание} & ---                      & \verb"\underline{Текст}"      \\
\emph{Выделение}          & ---                      & \verb"\emph{Текст}"           \\ \hline
\multicolumn{3}{|c|}{\bf Семейство (гарнитура)}                                      \\ \hline
\textrm{С засечками}      & \verb"{\rmfamily Текст}" & \verb"\textrm{Текст}"         \\
\textsf{Без засечек}      & \verb"{\sffamily Текст}" & \verb"\textsf{Текст}"         \\
\texttt{Равноразмерный}   & \verb"{\ttfamily Текст}" & \verb"\texttt{Текст}"         \\ \hline
\multicolumn{3}{|c|}{\bf Начертание}                                                 \\ \hline
\textup{Прямой}           & \verb"{\upshape Текст}"  & \verb"\textup{Текст}"         \\
\textit{Курсив}           & \verb"{\itshape Текст}"  & \verb"\textit{Текст}"         \\
\textsl{Наклонный}        & \verb"{\slshape Текст}"  & \verb"\textsl{Текст}"         \\
\textsc{Капитель}         & \verb"{\scshape Текст}"  & \verb"\textsc{Текст}"         \\ \hline
\multicolumn{3}{|c|}{\bf Насыщенность}                                               \\ \hline
\textmd{Средний (обычный)}& \verb"{\mdseries Текст}" & \verb"\textmd{Текст}"         \\
\textbf{Полужирный}       & \verb"{\bfseries Текст}" & \verb"\textbf{Текст}"         \\ \hline
\end{tabular}
\end{center}

При переходе от курсивного и наклонного шрифта к прямому при использовании команд-переключателей нужно ставить дополнительную команду \verb"\/" для увеличения пробела. Для команд с аргументами этого делать не нужно.

\begin{center}
\noindent\begin{tabular}{l|l}
\textbf{Набор}                                & \textbf{Эффект}                        \\ \hline
\verb"Упал {\itshape столб} Васе на голову"   & Упал {\itshape столб} Васе на голову   \\
\verb"Упал {\itshape столб\/} Васе на голову" & Упал {\itshape столб\/} Васе на голову \\
\verb"Упал \textit{столб} Васе на голову"     & Упал \textit{столб} Васе на голову     \\
\end{tabular}
\end{center}

\bigskip

<<В наследство>> от \LaTeX 2.09 и для совместимости с ранее набранными документами доступны команды-переключатели 
\begin{center}
\verb| \rm   \sf   \tt   \it   \sl   \em   \bf|
\end{center}
однако пользоваться ими \emph{настоятельно не рекомендуется}! 

\medskip

Помимо прочего, в \LaTeXe они работают не всегда корректно:

\begin{center}
\verb"{\rm Roman {\it Italic {\bf Bold-Italic}}}"

{\rm Roman {\it Italic {\bf Bold-Italic}}}

\bigskip

\verb"{\rmfamily Roman {\itshape Italic {\bfseries Bold-Italic}}}"  

{\rmfamily Roman {\itshape Italic {\bfseries Bold-Italic}}}

\bigskip

\verb"\textrm{Roman \textit{Italic \textbf{Bold-Italic}}}"

\textrm{Roman \textit{Italic \textbf{Bold-Italic}}}

\end{center}



\newpage

\textbf{4. Размеры шрифтов}

\bigskip

\begin{center}
\noindent\begin{tabular}{|l|l|}
\hline
\bfseries     Размер              &\bfseries Команда    \\ \hline
\tiny         Самый мелкий        &\verb"\tiny"         \\ \hline
\scriptsize   Мелкий (как индекс) &\verb"\scriptsize"   \\ \hline
\footnotesize Мелкий (как сноска) &\verb"\footnotesize" \\ \hline
\small        Маленький           &\verb"\small"        \\ \hline
              Обычный             &\verb"\normalsize"   \\ \hline
\large        Увеличенный         &\verb"\large"        \\ \hline
\Large        Большой             &\verb"\Large"        \\ \hline
\LARGE        Очень большой       &\verb"\LARGE"        \\ \hline
\Huge         Громадный           &\verb"\Huge"         \\ \hline
\end{tabular}
\end{center}

\newpage

\textbf{5. Абзацы}

\bigskip

\begin{center}
Центрированный абзац ---\\ окружение \texttt{center}.\\[2mm]
\verb"\begin{center} ... \end{center}"
\end{center}

\begin{flushleft}
Абзац, \\выровненный по левому краю ---\\ окружение \texttt{flushleft}.\\[2mm]
\verb"\begin{flushleft} ... \end{flushleft}"
\end{flushleft}

\begin{flushright}
Абзац, \\выровненный по правому краю ---\\ окружение \texttt{flushright}.\\[2mm]
\verb"\begin{flushright} ... \end{flushright}"
\end{flushright}

\bigskip

\centerline{Одна строчка по центру --- команда \texttt{\symbol{`\\}centerline\{\ldots\}}}

\bigskip

{\raggedright
Абзац вообще безо всякого выравнивания и без переносов --- команда \texttt{raggedright}. Обратите внимание, как она задается! Неочевидное положение фигурной скобки, закрывающей группу с форматируемым абзацем, связано с особенностью трансляции текста \TeX ом в dvi-файл. Дело в том, что \TeX\ форматирует абзац только после того, как встретит в~тексте пустую строку. Если не включить в группу эту пустую строку, команда будет фактически проигнорирована.

}

\newpage

\textbf{6. Разрывы строк и абзацев. Переносы}


\bigskip

Основной способ разорвать абзац и начать новый --- это сделать в исходном файле пустую строку. \par Того же эффекта можно добиться, применяя команду \verb"\par".

\bigskip

Принудительно разорвать строку, не начиная нового абзаца,\\ можно командой \verb"\\".

\bigskip

Принудительно разорвать строку, не начиная нового абзаца,\linebreak можно и командой \verb"\linebreak", но она <<тянет>> строки.

\bigskip

В необязательном аргументе команды \verb"\\[0.3cm]" можно указать,\\[0.3cm] какой вертикальный пробел нужно вставить.

\bigskip

Если написать \verb"\\*" или \verb"\\*[0.3cm]", это будет означать, что на оборванной строке нельзя оканчивать страницу.

\bigskip

Необязательный аргумент команды \verb"\linebreak[X]" указывает <<желательность>> разрыва строки. $X=0$ писать бессмысленно, а $X=4$ эквивалентно команде без аргумента.

\bigskip

Команды \verb"\nolinebreak" и \verb"\nolinebreak[X]" действуют противоположно.

\vspace{1cm}

Ручная установка возможных мест переноса:

\begin{center}
\verb"до\-сто\-при\-ме\-ча\-тель\-ность"
\end{center}

Запрещение переноса слова:

\begin{center}
\verb"\mbox{достопримечательность}"
\end{center}

Запрещение переноса формулы:

\begin{center}
\verb"{$x+y+z+t+u+v+w=0$}"
\end{center}

Обучение \TeX а переносить слова (в преамбуле):

\begin{center}
\verb"\hyphenation{дезоксирибонуклеиновая кис-ло-та}"
\end{center}

\TeX{} не умеет сам переносить слова с диакритическими знаками.


\newpage

\textbf{7. Специальные абзацы}

\bigskip

Для оформления длинных цитат используются окружение \texttt{quote} (также \texttt{quotation} для очень длинных цитат, состоящих из нескольких абзацев), ведь как писал сам Дональд Кнут\footnote{Дональд Кнут (род. 10 января 1938 г.) --- американский учёный, профессор Стэнфордского и других университетов в разных странах, преподаватель и идеолог программирования, создатель настольных издательских систем \TeX{} и METAFONT.}:
\begin{quote}
Пользователей \TeX а можно называть \TeX никами, но только те, кто используют возможности \TeX а по максимуму, достойны называться \TeX пертами.
\end{quote}

\vspace{2cm}

Для набора стихотворений предназначено специальное окружение \texttt{verse}:

\begin{verse}
Солнце взошло над Фудзиямой,\\
Сакура расцвела на древних склонах,\\
А кто-то сочинил эти дурацкие стихи для демонстрации возможностей ТеХа\ldots
\end{verse}


\newpage

\textbf{8. Интервалы, отступы и разрывы страниц}

\bigskip

Интервалы и отступы являются параметрами абзацев, указывать их надо либо в преамбуле документа, либо в документе, оформляя так же, как упоминавшуюся выше команду \verb"\raggedright". Приведённые ниже команды позволяют создавать абзацы специальной формы, что бывает полезно, например, при организации обтекания рисунков.

\bigskip

\begin{center}
\begin{tabular}{lp{8cm}}
\bfseries Команда         & \bfseries Значение                                      \\ \hline
\verb"\parindent=X"       & Отступ первой (красной) строки                          \\
\verb"\noindent"          & Абзац без отступа (красной строки)                      \\
\verb"\parskip=X"         & Вертикальный промежуток между абзацами                  \\ \hline
\verb"\hangindent=X"      & Отступ второй и последующей строк от левого края        \\
\verb"\hangindent=-X"     & Отступ второй и последующей строк от правого края       \\
\verb"\hangafter=X"       & hangindent будет работать со строки номер Х             \\
\verb"\hangafter=-X"      & hangindent будет работать со строками 1--Х              \\ \hline
\verb"\enlargethispage{X}"& Увеличить или уменьшить высоту страницы на Х
                            (pt, cm, mm\ldots\ или \verb"\baselineskip", т.е. строк)\\ \hline
\verb"\smallskip"         & Маленький промежуток (полстроки)                        \\
\verb"\medskip"           & Средний промежуток (строка)                             \\
\verb"\bigskip"           & Большой промежуток (2 строки)                           \\ \hline
\verb"\vspace{X}"         & Вертикальный промежуток заданного размера               \\
\verb"\hspace{X}"         & Горизонтальный промежуток заданного размера             \\
\end{tabular}
\end{center}


Управление разрывом страниц:

\begin{center}
\begin{tabular}{|l|l|}
\hline
\bfseries Команда      &\bfseries Значение                                  \\ \hline
\verb"\newpage"        & Принудительный разрыв страницы                     \\
\verb"\clearpage"      & Действует примерно так же, как \verb"\newpage"     \\
\verb"\cleardoublepage"& Новая страница с нечетным номером (не всегда)      \\
\verb"\pagebreak"      & Разрыв страницы с выравниваем по высоте            \\
\verb"\pagebreak[X]"   & $X$ --- желательность разрыва страницы (0\ldots4)  \\
\verb"\nopagebreak"    & Запрет на разрыв страниц в данном абзаце           \\
\verb"\nopagebreak[X]" & $X$ --- нежелательность разрыва страницы (0\ldots4)\\ \hline
\end{tabular}
\end{center}


\newpage

\textbf{9. Единицы длины}

Многие параметры, используемые в \TeX овских командах, являются единицами длины. В таблице представлены наиболее часто применяемые единицы длины.

\begin{center}
\begin{tabular}{|l|l|l|}
\hline
\textbf{Обозначение} & \textbf{Название} & \textbf{Пояснение}    \\ \hline
mm                   & миллиметр         &                       \\
cm                   & сантиметр         & 10 миллиметров        \\
in                   & дюйм              & 25.4 мм               \\
pt                   & пункт             & $\approx 0.35$ мм     \\
dd                   & пункт Дидо        & $\approx 1.07$ пункта \\
pc                   & пика              & 12 пунктов            \\
cc                   & цицеро            & 12 пунктов Дидо       \\ \hline
em                   & \multicolumn{2}{|l|}{ширина буквы <<M>> текущего шрифта}    \\
ex                   & \multicolumn{2}{|l|}{высота буквы <<x>> текущего шрифта}    \\ \hline
\end{tabular}
\end{center}

\bigskip


\textbf{10. Линейки}

Линейками в типографском деле называют любые черные прямоугольники (в том числе и линии). В \TeX е для вывода линеек предусмотрено несколько команд:\\
Команда \verb"\rule[X]{W}{H}" выводит прямоугольник высоты $H$, ширины $W$, сдвинутый по вертикали на $X$. (\emph{Примечание.} \textsl{Такая линейка трактуется \TeX ом как буква.})

Примеры: \rule{1em}{1.5ex} ~ \rule{2mm}{10pt} ~ \rule[-5pt]{2em}{.5ex}

\bigskip

Команда \verb"\vrule" задает вертикальную линейку в тексте \vrule ~по высоте равной высоте строки.

\smallskip

\hrule

\smallskip

Команда \verb"\hrule" задает линейку МЕЖДУ абзацами. Эту команду рекомендуется обрамлять командами \verb"\smallskip", чтобы линия не сливалась с буквами.

\bigskip

Эти команды можно слегка модифицировать: указать в явном виде высоту горизонтальной и ширину вертикальной линейки:
\begin{center}

\verb"\hrule height 2mm"

\smallskip

\hrule height 2mm

\end{center}


\begin{center}

\verb"\vrule width 0.1mm" \ \vrule width 0.1mm \\
\verb"\vrule width 0.5mm" \ \vrule width 0.5mm \\
\verb"\vrule width 1.0mm" \ \vrule width 1.0mm \\
\verb"\vrule width 2.0mm" \ \vrule width 2.0mm \\

\end{center}








\end{document} 